\documentclass{article}
\usepackage[T2A]{fontenc}
\usepackage[utf8]{inputenc}
\usepackage[english, russian]{babel}
\usepackage{hyperref}
\usepackage{amsmath}
\usepackage{amsfonts}

\setlength{\parindent}{0pt}

\begin{document}

\tableofcontents

\section{Линейная Алгебра}

\subsection{Линейное (векторное) пространство}

Линейное пространство -- это набор элементов (векторов), для которых определена операция сложения и умножения на число. Эти операции должны подчиняться определенному набору аксиом. \\

Детальная статья в Википедии (в которой в том числе перечислены все аксиомы): \href{https://ru.wikipedia.org/wiki/%D0%92%D0%B5%D0%BA%D1%82%D0%BE%D1%80%D0%BD%D0%BE%D0%B5_%D0%BF%D1%80%D0%BE%D1%81%D1%82%D1%80%D0%B0%D0%BD%D1%81%D1%82%D0%B2%D0%BE}{Векторное пространство}. \\

Примеры линейных (векторных) пространств:

\begin{itemize}
	\item Множество векторов на плоскости.
	\item Множество всех матриц размерности $m \times n$.
	\item Множество всех многочленов степени не выше $n$: \\ $f(x) = a_0 + a_1 \cdot x + a_2 \cdot x^2 + \dots + a_n \cdot x^n$
\end{itemize}

\subsection{Линейная зависимость и независимость векторов}

Рассмотрим набор векторов $v_1, v_2, \dots, v_n$. Данный набор векторов является \textbf{линейно зависимым}, если существуют такие числа $a_1, a_2, \dots, a_n$, что хотя бы одно из этих чисел не равно нулю, и при этом выполнено равенство

$$ a_1 \cdot v_1 + a_2 \cdot v_2 + \dots + a_n \cdot v_n = 0 $$

Если же равенство выше равно нулю только в том случае, если все числа $a_1, a_2, \dots, a_n$ равны нулю, то данный набор векторов называется \textbf{линейно независимым}.

\subsection{Размерность линейного пространства. Базис.}

Рассмотрим линейное пространство $L$. Рассмотрим набор из $n$ векторов $$v_1, v_2, \dots, v_n$$ принадлежащих этому пространству. Предположим, что этот набор векторов является линейно независимым и при этом любой набор из $n + 1$ векторов из этого же пространства является линейно зависимым. В таком случае $L$ называется $n$-мерным векторным пространством, и размерность этого пространства $dim(L) = n$. \\

Вектора $v_1, v_2, \dots, v_n$ образуют \textbf{базис} этого линейного пространства. Любой вектор $u \in L$ можно единственным образом представить в виде линейной комбинации векторов базиса. \\

\subsection{Подпространство}

Множество векторов $$u_1, u_2, \cdots, u_m$$ принадлежащих $L$, образует \textbf{подпространство} $M$, если для этих векторов заданы те же операции сложения и умножения на число, что и в исходном пространстве, и при этом любой вектор $u$, который является результатом выполнения этих операций, также принадлежит $M$.

\subsection{Системы линейных уравнений}

Урок на Stepik: \href{https://stepik.org/course/2461/syllabus}{Существование решений систем линейных уравнений}. \\

\subsubsection{Частный случай. Число уравнений равно числу неизвестных.}

Рассмотрим следующую систему линейных уравнений:

\[ \begin{array}{c}
	a_{11} x_{1} + a_{12} x_2 + a_{13} x_3 = b_1 \\
	a_{21} x_{1} + a_{22} x_2 + a_{23} x_3 = b_2 \\
	a_{31} x_{1} + a_{32} x_2 + a_{33} x_3 = b_3 \\
\end{array} \]

В такой системе количество уравнений совпадает с количеством неизвестных. Запишем систему в следующем виде:

\[
	x_1 \cdot \begin{pmatrix} a_{11} \\ a_{21} \\ a_{31} \end{pmatrix} + 
	x_2 \cdot \begin{pmatrix} a_{12} \\ a_{22} \\ a_{32} \end{pmatrix} + 
	x_3 \cdot \begin{pmatrix} a_{13} \\ a_{23} \\ a_{33} \end{pmatrix} =
	\begin{pmatrix} b_1 \\ b_2 \\ b_3 \end{pmatrix} 
\]

В таком виде задачу о нахождении решения данной системы можно рассматривать как задачу о представлении вектора $\mathbf{b}$ в виде линейной комбинации векторов $\mathbf{a_1}$, $\mathbf{a_2}$ и $\mathbf{a_3}$. \\

Если вектора $\mathbf{a_1}$, $\mathbf{a_2}$ и $\mathbf{a_3}$ образуют базис, то решение у такой системы существует при любом векторе $\mathbf{b}$, причем такое решение будет единственным. Если же эти вектора базис не образуют, то решение у системы будет существовать только в том случае, если вектор $\mathbf{b}$ будет принадлежать подпространству, пораждаемому векторами $\mathbf{a_1}$, $\mathbf{a_2}$ и $\mathbf{a_3}$, причем решений в таком случае будет бесконечно много. \\

Аналогичные утверждения верны и для системы линейных уравнений с $n$ уравнениями и $n$ неизвестными. \\

\subsubsection{Общий случай}

Рассмотрим теперь более общий случай. А именно, рассмотрим систему, состояющую из $n$ линейных уравнений с $m$ неизвестными:

\[ \begin{array}{c}
	a_{11} x_{1} + a_{12} x_2 + \cdots + a_{1m} x_m = b_1 \\
	a_{21} x_{1} + a_{22} x_2 + \cdots + a_{2m} x_m = b_2 \\
	\cdots \\
	a_{n1} x_{1} + a_{n2} x_2 + \cdots + a_{nm} x_m = b_n \\
\end{array} \]

Перепишем систему в следующем виде:

\[
	x_1 \cdot \begin{pmatrix} a_{11} \\ a_{21} \\ \cdots \\ a_{n1} \end{pmatrix} + 
	x_2 \cdot \begin{pmatrix} a_{12} \\ a_{22} \\ \cdots \\ a_{n2} \end{pmatrix} +
	\cdots +
	x_m \cdot \begin{pmatrix} a_{1m} \\ a_{2m} \\ \cdots \\ a_{nm} \end{pmatrix} =
	\begin{pmatrix} b_1 \\ b_2 \\ \cdots\\  b_n \end{pmatrix} 
\]

В таком виде задачу о нахождении решения для данной системы уравнений можно рассматривать как задачу о представлении вектора $\mathbf{b}$ в виде линейной комбинации векторов $\mathbf{a_1}, \mathbf{a_2}, \cdots, \mathbf{a_m}$, каждый из которых является элементом $n$-мерного линейного пространства. \\

Рассмотрим линейное подпространство минимальной размерности, которое содержит все эти $m$ векторов. Такое подпространство также называется линейной оболочкой, образуемой данными векторами. Размерность такого подпространства (линейной оболочки) называется \textbf{рангом} системы линейных уравнений. \\

Касательно существования решения для системы таких уравнений. Возможны два случая: 

\begin{itemize}
	\item Если вектор $b$ не принадлежит данной линейной оболочке, то решений у системы нет.
	\item Если вектор $b$ принадлежит данной линейной оболочке то, решение существует. При этом если $n = m$, то решение будет единственным, так как набор векторов $\mathbf{a_1}, \mathbf{a_2}, \cdots, \mathbf{a_m}$ будет образовывать базис. Если же число векторов больше, чем размерность линейной оболочки, то система будет иметь бесконечно много решений.
\end{itemize}

\subsection{Решение систем линейных уравнений. Метод Гаусса.}

Урок на Stepik: \href{https://stepik.org/lesson/9582/step/1?unit=23533}{Решение систем линейных алгебраических уравнений. Метод Гаусса}. \\

Основная идея метода Гаусса заключается в том, чтобы c помощью операций сложения и умножения на число последовательно исключать переменные, приводя матрицу коэффициентов к треугольному (диагональному виду). Имея матрицу в таком виде, можно затем последовательно найти значения всех неизвестных. \\

Статья и примеры в wiki: \href{https://en.wikipedia.org/wiki/Gaussian_elimination}{Gaussian Elimination}. (в том числе используется для того, чтобы найти ранг и определитель матрицы).

\subsection{Евклидово пространство}

\subsubsection{Скалярное произведение}

Для двух векторов $u, v$, принадлежащих некоторому линейному пространству $L$, скалярным произведением называется операция, которая этим двум векторам сопоставляет некоторое вещественное число: $$(u, v) = c: \ c \in \mathbb{R}.$$

При этом такая операция должна удовлетворять 4-м аксиомам:

\begin{enumerate}
	\item $(x, y) = (y, x)$
	\item $(\lambda x, y) = \lambda \cdot (x, y) \ \forall \lambda \in \mathbb{R}$
	\item $(x_1 + x_2, y) = (x_1, y) + (x_2, y)$
	\item $(x, x) \ge 0$
\end{enumerate}

Линейное (векторное) пространство с введенной на нем вышеописанной операцией скалярного произведения, называется \textbf{Евклидовым пространством}. \\

Для векторов $x = (x_1, x_2, \dots, x_n), y = (y_1, y_2, \dots, y_n) \in \mathbb{R}^n$ примером скалярного произведения может выступать сумма произведений их координат:

$$ (u, v) = x_1 \cdot y_1 + x_2 \cdot y_2 + \dots + x_n \cdot y_n$$

\subsubsection{Угол между векторами, длина вектора}

Для векторов на плоскости скалярное произведение можно ввести следующим образом:

$$(x, y) = |x| \cdot |y| \cdot \cos(x, y)$$

Выразим отсюда косинус угла между векторами:

$$ \cos(x, y) = \frac{(x, y)}{|x| \cdot |y|} = \cos (\alpha) $$

Из этого выражения получим, что \textbf{угол между векторами} можно найти следующим образом:

$$ \alpha = \arccos\left(\frac{(x, y)}{|x| \cdot |y|}\right) $$

Рассмотрим теперь выражение для скалярного произведения, в котором $y = x$.

$$ (x, x) = |x| \cdot |x| \cdot \cos(x, x) = {|x|}^2 $$

Отсюда получим, что \textbf{длина вектора} $x$ есть

$$ |x| = \sqrt{(x, x)} $$

То есть, если мы знаем, чему равно скалярное произведение, то мы можем найти угол между векторами, а также длину вектора. \\

Данные понятия можно обобщить на случай произвольного векторного пространства. А именно, длину произвольного вектора $x$ можно определить как квадратный корень из скалярного произведения этого вектора на самого себя:

$$ |x| = \sqrt{(x, x)} $$

Угол $\phi$ между произвольными векторами $x, y$ есть

$$ \phi = \arccos\left(\frac{(x, y)}{|x| \cdot |y|}\right) $$

Для векторов на плоскости скалярное произведение будет равно нулю, если векторы ортогональны, то есть $\phi = \frac{\pi}{2}$. Два произвольных вектора будем называть \textbf{ортогональными}, если их скалярное произведение равно нулю.

\subsection{Операторы и базис}

Урок на Stepik: \href{https://stepik.org/lesson/9584/step/1?unit=23534}{Ортогональный базис}.

\subsubsection{Ортонормированный базис}

Рассмотрим набор векторов $\{e_1, e_2, \dots e_n\}$ таких, что

\begin{enumerate}
	\item $(e_i, e_j) = 0 \ \forall i, j: \ i \ne j$.
	\item $(e_i, e_j) = 1 \ \forall i, j: \ i = j$.
\end{enumerate}

Такой набор векторов называется ортонормированным набором векторов в линейном пространстве со скалярным произведением. \\

Уроки на Stepik:

\begin{itemize}
	\item \href{https://stepik.org/lesson/9584/step/6?unit=23534}{Как произвольный базис преобразовать в ортонормированный}
	\item \href{https://stepik.org/lesson/9584/step/13?unit=23534}{Метод наименьших квадратов}
\end{itemize}

\subsection{Линейные операторы}

Оператор $A: \ A \cdot x = y$ называется линейным оператором, если выполнены следующие аксиомы:

\begin{enumerate}
	\item $A \cdot (x_1 + x_2) = A \cdot x_1 + A \cdot x_2$
	\item $A (\lambda x) = \lambda \cdot Ax$
\end{enumerate}

Пример линейного оператора в двумерном пространстве - оператор повотора на угол $\alpha$:

\[
	\begin{pmatrix}
		\cos \alpha &  -\sin \alpha \\
		\sin \alpha & \cos \alpha
	\end{pmatrix}
\]

Для того, чтобы задать оператор, достаточно выяснить, как этот оператор действует на базисные вектора линейного пространства.

\subsubsection{Ядро и образ оператора}

- Ядро оператора - это все вектора, которые данный оператор обращает в нулевой вектор. \\
- Образ оператора - это множество всех ненулевых векторов, которые получаются в результате действия данного оператора. \\

\subsubsection{Собственные числа и собственные векторы}

$u$ - собственный вектор линейного оператора $A$, если $Au = \lambda u$. $\lambda$ - собственное число оператора. \\

- \href{https://en.wikipedia.org/wiki/Eigenvalues_and_eigenvectors}{Eigenvalues and eigenvectors} \\
- характеристический многочлен и характеристическое уравнение \\
- следствие из основной теоремы алгебры

\subsection{Матрицы. Произведение. Определитель и ранг.}

- Урок на Stepik: \href{https://stepik.org/lesson/44077/step/1?unit=21901}{Определитель матрицы}. \\
- Статья в Wiki: \href{https://en.wikipedia.org/wiki/Determinant}{Determinant}. \\
- Произведение матриц: \href{https://en.wikipedia.org/wiki/Matrix_multiplication}{Matrix multiplication}. \\
- Обшая статья про матрицы: \href{https://en.wikipedia.org/wiki/Matrix_(mathematics)}{Matrix} \\
- Ранг матрицы: \href{https://en.wikipedia.org/wiki/Rank_(linear_algebra)}{} \\

Определитель матрицы - это некоторое число, которое характеризует данную матрицу. \href{https://en.wikipedia.org/wiki/Leibniz_formula_for_determinants}{Формальное определение определителя (формула Лейбница)}.

\subsection{Уравнение прямой. Пересечение прямых.}

\href{https://portal.tpu.ru/SHARED/p/PEG/page_2/laag/Tab1/LAAG_Lecture-12.pdf}{Лекция: прямая на плоскости}

\section{Математический анализ}

\subsection{Вещественные числа}

Урок на Stepik: \href{https://stepik.org/lesson/28445/step/3?unit=9589}{Вещественные числа}. \\

Введем понятие вещественных чисел аксиоматически. Вещественные числа, это такие числа, которые удовлетворяют определенному набору аксиом. \\

\textbf{Первый набор аксиом} касается алгебраической структуры чисел. Рассмотрим две операции -- операции сложения и умножения на число, которые любой паре вещественных чисел сопоставляют вещественное число: $\mathbb{R} \times \mathbb{R} \to \mathbb{R}$. Такие операции должны удовлетворять 9-ти аксиомам:

\begin{itemize}
	\item Ассоциативность и коммутативность по сложению и умножению.
	\item Существование нейтрального и обратного элемента (для сложения и умножения).
	\item Дистрибудивность.
\end{itemize}

\textbf{Второй набор аксиом} касается порядковой структуры (отношение порядка). Для любых двух вещественных чисел $x$ и $y$ мы должны уметь определять, является ли $x$ меньшим, либо равным $y$ ($x \le y$). Такое отношение порядка должно удовлетворять следующему набору аксиом:

\begin{itemize}
	\item Рефлексивность: $x \le x$
	\item Транзитивность: $x \le y, \ y \le z \Rightarrow x \le z$
	\item Антисимметричность $x \le y, \ y \le x \Rightarrow x = y$
	\item Любые два вещественных числа должны быть сравнимы, то есть $\forall x, y$ можно однозначно сказать, верно ли, что $x \le y$
	\item Если $x \le y$, то $x + z \le y + z$
	\item Если $0 \le x$ и $0 \le y$, то $0 \le xy$
\end{itemize}

\textbf{Аксиома полноты:} $\forall A, B \subset \mathbb{R}$ и $\forall x,y: x \in A, \ y \in B$ и при этом $x \le y$, найдется $z \in R$, такое, что $x \le z \le y$.

\subsection{Последовательности. Предел последовательности.}

\textbf{Последовательностью} будем называть произвольное отображение из множества натуральных чисел в множество вещественных чисел: $\mathbb{N} \to \mathbb{R}$. Примеры последовательностей:

\begin{itemize}
	\item $x_n = n^2$
	\item $x_n = \frac{1}{n}$
\end{itemize}

Число $a$ называется \textbf{пределом} последовательности $\lim\limits_{n \to \infty} x_n = a$ если $$ \forall \epsilon > 0 \ \exists N :  \ \forall n > N \ |x_n - a| < \epsilon.$$

\subsection{Функции. Предел функции. Непрерывность.}

\subsubsection{Функции. Предел функции}

Функцией $f : \left(\mathbb{E} \subset \mathbb{R}\right) \to \mathbb{R}$ будем называть некоторое правило, по которому каждому элементу множества $E$ ставится в соответствие некоторое вещественное число. \\

\textbf{Определение предела функции} (по Коши). Число $A$ называется пределом функции $f : \mathbb{E} \to \mathbb{R}$ в точке $a$, если 

$$ \forall \epsilon > 0 \ \exists \delta > 0 : \forall x \in E : x \ne a, |x - a| < \delta \implies |f(x) - A| < \epsilon $$

Определение на языке пределов: $A$ -- предел функции $f$ в точке $a$, если для любой последовательности $\{x_n\} \in E$, такой, что $x_n \ne a$ и $\lim\limits_{n \to \infty} x_n = a$ выполнено $\lim\limits_{n \to \infty} f(x_n) = A$. \\

\href{https://en.wikipedia.org/wiki/Limit_of_a_function}{Wiki: Limit of a function}.

\subsubsection{Непрерывность функции}

Пусть $f : \mathbb{E} \to \mathbb{R}$ и $a \in E$, причем $a$ -- предельная точка множества $E$. Тогда $f$ непрерывна в точке $a$ если

$$ \lim\limits_{x \to a} f(x) = f(a) $$ 

Эквивалентное определение. Функция $f$ непрерывна в точке $a$, если

$$ \forall \epsilon > 0 \ \exists \delta > 0 : |x - a| < \delta \ \implies |f(x) - f(a)| < \epsilon $$

\href{https://ru.wikipedia.org/wiki/%D0%9D%D0%B5%D0%BF%D1%80%D0%B5%D1%80%D1%8B%D0%B2%D0%BD%D0%B0%D1%8F_%D1%84%D1%83%D0%BD%D0%BA%D1%86%D0%B8%D1%8F}{Wiki: непрерывная функция.}

\subsection{Производные 1}

Материалы на Stepik:

\begin{itemize}
	\item \href{https://stepik.org/lesson/28380/step/1}{Дифференцируемость и производная}
	\item \href{https://stepik.org/lesson/28380/step/4}{Геометрический смысл производной}
	\item \href{https://stepik.org/lesson/28380/step/6}{Правила дифференцирования}
	\item \href{https://stepik.org/lesson/28380/step/8}{Производные основных элементраных функций}
	\item \href{https://stepik.org/lesson/28369/step/2}{Теоремы о среднем}
	\item \href{https://stepik.org/lesson/28370/step/1}{Производная и монотонность}
	\item \href{https://stepik.org/lesson/28381/step/1}{Правило Лопиталя}
\end{itemize}

Будем говорить, что функция $f$ дифференцируема в точке $x_0$ если существует такое число $k \in \mathbb{R}$, что

$$ f(x) = f(x_0) + k \cdot (x - x_0) + o(x - x_0) $$

То есть значение функции в точке $x_0$ можно посчитать так, как описано выше. \\

Производной функции $f$ в точке $x_0$ называется предел отношения приращения функции к приращению аргумента:

$$ f^{\prime}(x_0) = \lim\limits_{x \to x_0} \frac{f(x) - f(x_0)}{x - x_0} $$

Значение производной в заданной точке равно угловому коэффициенту касательной в этой точке (геометрический смысл производной). \\

\textbf{Теорема Ферма}. Пусть $f$ - функция, заданная на промежутке $<a, b> \ \to \mathbb{R}$ и дифференцируема в точке $x_0 \in (a, b)$. Пусть $f(x_0)$ - максимум или минимум функции $f$ в точке $x_0$. Тогда $f^{\prime}(x_0) = 0$. \\

\textbf{Условия монотонности функции.} Пусть $f: \ <a, b> \ \to \mathbb{R}$ непрерывна и дифференцируема на $(a, b)$. Тогда функция $f$ возрастает на $<a, b>$ тогда и только тогда, когда $f^{\prime}(x) \ge 0 \ \forall x \in (a, b)$. Если $f^{\prime}(x) > 0$, то функция строго возрастает.

\subsection{Производные 2}

Материалы на Stepik:

\begin{itemize}
	\item \href{https://stepik.org/lesson/32797/step/1}{Формула Тейлора}
	\item \href{https://stepik.org/lesson/28372/step/2}{Экстремумы функции}
\end{itemize}

\subsubsection{Разложение функции в ряд Тейлора}

Рядом Тейлора называется выражение вида $$ \sum_{n = 0}^{\infty} \frac{f^{(n)}(x_0)}{n!} (x - x_0)^n $$

Если для функции $f$ все её производные ограничены некоторой константой $M$, то 

$$ f(x) = \sum_{n = 0}^{\infty} \frac{f^{(n)}(x_0)}{n!} (x - x_0)^n $$

В частности $$e^x = \sum_{n = 0}^{\infty} \frac{x^n}{n!}$$

\subsubsection{Необходимые и достаточные условия экстремума функции}

Для того, чтобы точка $x_0$ являлась экстремумом функции $f$, необходимо, чтобы $f^{\prime}(x_0) = 0$. Но если последнее выполнено, то это не значит, что точка $x_0$ является экстремумом. Достаточные условия того, что точка $x_0$ - экстремум, такие:

\begin{itemize}
	\item Если $f^{\prime \prime}(x_0) < 0$, то $x_0$ - точка максимума.
	\item Если $f^{\prime \prime}(x_0) > 0$, то $x_0$ - точка минимума.
\end{itemize}

\subsection{Интегралы 1}

\begin{itemize}
	\item \href{https://stepik.org/lesson/28376/step/1}{Первообразная и неопределенный интеграл}
	\item \href{https://stepik.org/lesson/28377/step/1}{Действия с неопределенными интегралами}
	\item \href{https://stepik.org/lesson/28378/step/2}{Площади}
	\item \href{https://stepik.org/lesson/28379/step/1}{Определенный интеграл}
\end{itemize}

Функция $F(x)$ - первообразная функции $f(x)$, если $F^{\prime}(x) = f(x)$. \\

Неопределенным интегралом для функции $f$ называется множество всех первообразных этой функции:

$$ \int f(x) dx = F(x) + C $$

- \href{https://stepik.org/lesson/28376/step/3}{Таблица основных интегралов}. \\
- \href{https://stepik.org/lesson/28377/step/2}{Действия с неопределенными интегралами} \\
- \href{https://stepik.org/lesson/28379/step/2}{Определенный интеграл} \\

Рассмотрим функцию $f : [a, b] \to \mathbb{R}$. Определенным интегралом $f$ по отрезку $[a, b]$ назовем разность положительных и отрицательных состовляющих площадей функции. Обозначение: $$ \int_{a}^{b} f(x) dx $$

Формула Ньютона-Лейбница: $$ \int_{a}^{b} f(x) dx = F(b) - F(a) $$

\subsection{Интегралы 2}

- \href{https://stepik.org/lesson/35886/step/2}{Равномерная непрерывность} \\
- \href{https://stepik.org/lesson/28384/step/1}{Интегральные суммы}

\section{Теория вероятностей}

См. stepik-notes

\section{Комбинаторика}

См. stepik-notes

\section{Теория графов}

\subsection{Потоки и сети}

Урок на Stepik: \href{https://stepik.org/lesson/36127/step/10?unit=44097}{Потоки и сети}. \\

Под \textbf{сетью} будем понимать некоторый ориентированный граф $G(V, E, s, t, c)$, где

\begin{itemize}
	\item $V$ - множество вершин
	\item $E$ - множество ребер
	\item Существует вершина $s \in V$ такая, что $indeg(s) = 0$. Будем называть такую вершину \textbf{истоком}.
	\item Существует вершина $t \in V$ такая, что $outdeg(t) = 0$. Будем называть такую вершину \textbf{стоком}.
	\item Задана функция $c: E \to \mathbb{R^+}$, которая любому ребру $(x, y) \in E$ ставит в соответствие некоторое вещественное неотрицательное число $c(x, y)$. 
\end{itemize}

Функцию $c(x, y)$ будем называть \textbf{пропускной способностью} этой цепи. \\

Рассмотрим функцию $f$, которая каждому ребру графа $G$ ставит в соответствие вещественное неотрицательное число $f(x, y)$. Пусть для функции $f$ выполняются следующие ограничения:

\begin{itemize}
	\item Для всех ребер $(x, y) \in E$ верно, что $f(x, y) \le c(x, y)$.
	\item Для любой вершины $y$ отличной от $s$ и $t$ верно, что $$\sum_{x : (x, y) \in E} f(x, y) = \sum_{z : (y, z) \in E} f(y, z)$$.
\end{itemize}  

Тогда $f(x, y)$ будем называть \textbf{потоком}, проходящим через ребро $(x, y)$. \textbf{Величиной потока} будем называть величину $Q$, определяемую так: $$Q = \sum_{x: (s, x) \in E} f(s, x) $$

Рассмотрим разбиение множества всех вершин графа $G$ на два множества $S$ и $T$: $s \in S$ и $t \in T$. Рассмотрим множество ребер $R(S, T)$, исходящих из вершин множества $S$ и входящих в вершины множества $T$. Будем называть такое множество \textbf{разрезом}. \textbf{Пропускной способностью} $C(S, T)$ данного разреза будем называть следующую величину: $$ C(S, T) = \sum_{(x, y) \in R} c(x, y) $$

Заметим, что никакой поток $Q$ не может превосходить величины $C(S, T)$ никакого из разрезов $R(S, T)$. Рассмотрим разрез, в котором пропускная способность минимальна. \\

\textbf{Утверждение (Теорема Форда-Фалкерсона)}. Величина максимального потока в сети совпадает величиной пропускной способности минимального разреза. \\

\href{https://www.cs.princeton.edu/courses/archive/spr04/cos226/lectures/maxflow.4up.pdf}{maxflow.4up.pdf (Лекция про потоки)}.

\section{Другое}

- \href{https://ru.wikipedia.org/wiki/%D0%90%D1%80%D0%B8%D1%84%D0%BC%D0%B5%D1%82%D0%B8%D1%87%D0%B5%D1%81%D0%BA%D0%B0%D1%8F_%D0%BF%D1%80%D0%BE%D0%B3%D1%80%D0%B5%D1%81%D1%81%D0%B8%D1%8F}{Aрифметическая прогрессия} \\
- \href{https://ru.wikipedia.org/wiki/%D0%93%D0%B5%D0%BE%D0%BC%D0%B5%D1%82%D1%80%D0%B8%D1%87%D0%B5%D1%81%D0%BA%D0%B0%D1%8F_%D0%BF%D1%80%D0%BE%D0%B3%D1%80%D0%B5%D1%81%D1%81%D0%B8%D1%8F}{Геометрическая прогрессия} \\

\end{document}
















